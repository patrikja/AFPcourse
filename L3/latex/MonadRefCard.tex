% -*- latex -*-

\documentclass[times, 10pt,twocolumn,preprint]{article} 
\usepackage{times}
\usepackage[mathletters]{ucs}
\usepackage[utf8x]{inputenc}
\usepackage{latexsym}
%include lhs2TeX.fmt

\usepackage{tikz}
\usepackage{amsmath}
\usepackage{verbatim}
\usetikzlibrary{arrows,shapes}

\usepackage{amsthm}
\usepackage{natbib}



\newtheorem{theorem}{Theorem}
\newtheorem{lemma}[theorem]{Lemma}
\newtheorem{corollary}[theorem]{Corollary}
\theoremstyle{definition}
\newtheorem{definition}{Definition}


\newcommand{\prop}{\mathsf{prop}}
\newcommand{\LAMBDA}[2]{{\lambda #1.\,#2}}
\newcommand{\FORALL}[3]{{\forall #1 : #2.\,#3}}
\newcommand{\Subs}[3]{{#1}[{#2} \mapsto {#3}]}

\newcommand{\Assert}[3]{{#1 \vdash #2 : #3}}


\newcommand{\many}[1]{\overline{#1}}
\newcommand{\subR}[1]{{#1}_R}
\newcommand{\trans}[1]{[\![#1]\!]}
\newcommand{\bind}{\mathbin{>\!\!>\!\!=}}

% \newcommand{\tsort}[1]{\fbox{\ensuremath{#1}}}
\newcommand{\tsort}[1]{\underline{#1}}

\newcommand{\reducesto}[0]{\longrightarrow_\beta}
% \newcommand{\implies}[0]{\Rightarrow}

\newcommand{\para}[1]{
  \vspace{0.8cm}
  \paragraph{#1}}

\title{Monads reference card}
\author{Jean-Philippe Bernardy}

\begin{document}

\maketitle

% For every picture that defines or uses external nodes, you'll have to
% apply the 'remember picture' style. To avoid some typing, we'll apply
% the style to all pictures.
\tikzstyle{every picture}+=[remember picture]

% By default all math in TikZ nodes are set in inline mode. Change this to
% displaystyle so that we don't get small fractions.
\everymath{\displaystyle}

\tikzstyle{na} = [baseline=-.8mm]

\para{Syntax}
\begin{tabular}{lc}
monad component  & DSL application \\
\hline
$m :: \star \to \star$ & Expressions parameterized on return type \\
$return :: a \to m \, a$ & constant expression \\
$(\bind{}) :: m \, \tikz[na]{ \node[fill=blue!20] (a1) {$a$}; } \to (\tikz[na]{ \node[fill=blue!20] (a2) {$a$}; }  \to m\,b ) \to m\,b$ & bind an $a$ returned by the lhs into the rhs \\
\end{tabular}

\begin{tikzpicture}[overlay]
        \path[->] (a1) edge [bend right=90] node [auto,swap] {\small bind} (a2);
\end{tikzpicture}

\para{Laws}
\begin{tabular}{ccc}
name & law & a DSL aspect \\
\hline
left identity & $return \, a \bind{} (\LAMBDA{x}{m\,x}) \equiv m \, a $ & inlining/factorizing a constant \\
right identity & $m \bind{} (\LAMBDA{x}{return\,x})\equiv m$ &
removal/introduction of useless return \\
associativity & $(m \bind{} f) \bind{} g \equiv  m \bind{} (\LAMBDA x {f \,x \bind{} g})$ & extension/shrinking of scope \\ 
\end{tabular}

\para{``do''}
\begin{tabular}{rl@{\qquad\qquad}|@{\qquad\qquad}rl}
do & $x \leftarrow α$      & $ α$ & $\bind{} \LAMBDA x {} $\\
   & $y \leftarrow β$      & $ β$ & $\bind{} \LAMBDA y {} $\\
   & $γ$ &  $γ$ & \\
\end{tabular}
\begin{itemize}
\item parentheses are not needed
\item $x$ may appear in $\gamma$
\end{itemize}

\para{Comprehensions}
\begin{tabular}{rl@{\qquad\qquad}|@{\qquad\qquad}rl}
[       & $γ$  &  & \\
$|$     & $x \leftarrow α$      & $ α$ & $\bind{} \LAMBDA x {} $\\
$,$     & $y \leftarrow β$      & $ β$ & $\bind{} \LAMBDA y {} $\\
]       &  &  return $γ$ & \\
\end{tabular}


\begin{itemize}
\item $\bind{}$ can be used to ``flatten'' levels of the monad.
\item $join :: m (m\, a)  \to m\, a$
\item $join \, xs = xs \bind{} id$
\end{itemize}


\end{document}
